\chapter{What is Terrorism?}

\section{The taxonomy of terrorism}

From such high profile incidents as the September 11th attacks on the US eastern coast, to Beslan in Russia in 2004 to the attacks on the World trade centre in 2001, to more recent attacks in San Bernadino and the Nice attacks in 2016, terrorism is now perceived as a major threat to many countries. Terrorism though is not a new phenomenon and has existed since ancient times from the Sicarii \citep{horsley1979sicarii} to the Assassins \citep{sloan2009historical}. Later terrorism took on more sophisticated methodologies from the gunpowder plot in the 16th century \citep{fraser2010gunpowder}, through to the nationalist and politically motivated groups of the 20th century, on to the religiously motivated groups of the twenty first century.

Terrorism even from the small number of examples cited previously can be characterized as being adaptive and has adopted new technologies, tactics, aims and ideologies. Due to the large scale increase in the adoption of advanced communications the means of disseminating it's message to propagandise  recruits, disseminate technology and aid in the exchange of information between members and between allied groups, its prevalence has grown considerably.

Terrorism's commonly accepted definition is the 'utilization and/or the perceived menace of violence whose aim is to further a political agenda'. While there is no universally agreed definition of terrorism \citep{ruby2002definition}, with various governments, institutions, law enforcement agencies and legal entities having different definitions of terrorism. For instance the U.S department of defence defines terrorism as \citep{pub1998pub}:
"politically motivated violence perpetrated against non combatant targets by sub-national groups or clandestine agents."

While NATO defines terrorism as \citep{chase2013defining}:
"The unlawful use or threatened use of force or violence against individuals or property in an attempt to coerce or intimidate governments or societies to achieve political, religious or ideological objectives".
The most commonly agreed tenets of terrorism can be summarized as such:
\begin{enumerate}
\item It is the utilization of violence in order to coerce a government into a political, religious or ideological shift in policy \citep{chase2013defining}.
\item It is committed by non-state actors (or in the case of state sponsored terrorism) or the indirect or direct usage of a states military or para-military forces.
\item It is aimed at influencing society as a whole besides the immediate victims of an act of terrorism. The immediate victims can be seen as a surrogate for the state, ideology or religion under attack, through the use of media to trumpet their (the terrorists) message,\citep{el2014terrorist}.
\item Another important tenet of terrorism is that when considered in the terms of international law as outlined by the Geneva and Hague conventions, they are considered to be '\textit{mala prohibita}' (a crime that is illegal according to legislation) and \textit{'mala in se'} (a crime that is morally wrong), \citep{ganor2002defining}.
\item To the terrorist, their acts are considered to be \textit{'Jus ad bellum'} the idea of a just war, being either legitimized from a religious, political or constitutional point of view \citep{kennedy1999one}. 
\end{enumerate}

\section{Creating the conditions for terrorism?}

For terrorism to occur, a number of criteria must be met \citep{crenshaw1981causes}. The first of these is that an identifiable grievance must exist amongst a detectable section of the populace. This can often be an ethnic minority who feel discriminated against by the majority as in the case of Irish nationalists during the troubles in Northern Ireland \citep{moxon1983nation}, Basques in Spain \citep{llera1993eta}, or dissident Quebecois \citep{ross1995rise} who wanted to separate from Canada. However a dissatisfied minority is often not a sufficient condition to establish a terrorist group. Not everyone who is subjected to deprivation and discrimination will turn to terrorism or neither all persons who turn to terrorism will be from a persecuted background. 

Numerous cases of people turning to terrorism who were not subject to deprivation exist, examples being the red brigades of Germany, Italy and Japan of the 1970's \citep{zwerman2000disappearing}. Therefore it is not a case of actual deprivation or discrimination but a case of perceived deprivation, with the government to blame for this deprivation or discrimination that is responsible for radicalization.
\\
Context is another important causal factor, especially when it affects an elite and not the population as a whole.  The elite constituting a well educated, middle class and young disaffected cohort of the population, who see the only chance of affecting change through drastic measures such as terrorism \citep{ronchey1979guns}.
\\
A third factor necessary for (the exclusive use of as opposed to the use of terrorism as part of an overall insurgency) terrorism is the intersection of not only elite discontent and dissatisfaction but with indifference amongst the population. 

The final causal factor identified by Crenshaw \citep{crenshaw1981causes} is the idea of a precipatory event involving a demonstrative and impulsive use of force that occurs directly preceding the outbreak of terrorism. Examples of this would be the killing of Beno Ohnesorg at the hands of the West German police in 1968 led to the development of the RAF (Red Army Faction) \citep{jazic2013rise}.

\section{Why terrorist groups commit terrorist acts?}

Terrorists commit terrorist act's no only as an exclusive tactic but also as part of wider struggle or insurgency against a state. As such terrorist acts can be used along with guerilla warfare within an insurgency against a government as a tactic. Were terrorism diverges from guerilla warfare is that, were guerilla warfare targets military forces, terrorism targets civilians. Often insurgents will use both and legitimize the use of terrorism by considering it to be a form of politically motivated violence \citep{ganor2002defining}. Terrorists also see terrorism as a mechanism of radicalising the populace. They believe this to either occur through the use of sensational acts of terrorism to alight an insurgent spirit in the people or through over zealous responses of governments to terrorist acts which will cause large discontent in the populace and sympathy for the terrorists \citep{jenkins1985international}. A proponent of such a tactic has been the FARC group in Colombia who has utilized terrorism as part of a wider insurgency strategy  \citep{wickham1990terror},\citep{marks2002colombian}.
\\
Terrorists may also aim to impose an economic cost on a country. Acts of terrorism have a direct economic effect by discouraging DFI (Direct Foreign Investment) through  redirecting public funds to the security forces or through the impediment of foreign trade \citep{sandler2008economic}. While these consequences have a local economic effect terrorism also can have global consequences. In the aftermath of the September 11th attacks, saw both a substantial transitory lowering of demand but also a long term negative need or demand \citep{ito2005assessing} on US airline use.
\\
Other motivating factors for terrorists to carry out terrorist acts include the weakening of key infrastructure to create mistrust in government amongst the populace to provide for its citizens or to put pressure on a government to release prisoners  \citep{TerrorismTypes2016}. In doing this, they disrupt the processes of government. Doing so they demoralize officials of the state without impacting directly upon the public. 

\section{The strategic aims of terrorism}
\label{sec:chap1strataimsterr}

Related to types of terrorism are the strategic goals which lie at the aim of a terrorist campaign \citep{kydd2006strategies}, these are:
\begin{enumerate}
\item Attrition. This is a strategy employed by a group to persuade an opponent to give up due to the considerable cost of continuing their current policy. It is also based on the premise that the terrorist organization can sustain losses at higher rates than their opponents.
\item Intimidation. Intimidation is a control strategy, where the terrorists aim to persuade the populace that the terrorist group is of ample strength so as to punish a lack of loyalty to the group and also the government is not of sufficient strength to protect them, thereby subjugating the people, to assert control over the people.
\item Provocation. This is the use of terrorist groups to provoke responses with an indiscriminate unmeasured response, thereby instilling a sense of injustice in the populace which leads to a radicalization of the populace and support for the terrorist group.
\item Spoiling. A spoiling strategy is an attempt to dissuade a government from dialogues with groups whose cause may be allied with their own (the terrorist group), by launching spoiling attacks which may sabotage a peace/reconciliation process. 
\item Outbidding. This is the process of the use of violence to convince the general population that they posses greater steadfastness and will than their opponents and rivals.
\end{enumerate}

While these terms provide a useful categorization of strategies, it should be remembered that they are not mutually exclusive and groups will often indulge in usage of multiple different strategies to obtain their goals.

\section{Classifying the different types of terrorism}

Beginning in the 1970's researchers began to distinguish between the different types of terrorism based upon the act in order to better understand the subject, due to the perceived increase in threat of terrorism \citep{badey1998defining}. A clear consensus has developed on the different types, these are:
\begin{enumerate}
\item State terrorism. State terrorism can be defined as the use of force (or threat of force) with the aim of coercion and intimidation of a populace by state actors. This is either carried out directly by state actors (police, military) or carried out indirectly by agents acting on behalf of the state, which can be seen as a proxy for the state. A very early example of state terrorism would be revolutionary France in the late 18th century, where a revolutionary dictatorship who seized power after the fall of the monarchy, slaughtered tens of thousands who they thought may oppose the regime. Similarly the strongly authoritarian regimes of Nazi Germany \citep{gibbs1989conceptualization}, Maoist China and Stalinist Russia \citep{blakeley2009state} deployed state terrorism but this time on an industrial scale and killed millions. Later in the 20th century, Military juntas utilized  state terrorism as part of strategy to suppress populist left leaning or communist liberation movements, who themselves often employed terrorism. 

This use of state terrorism as a method of affecting a counter insurgency strategy was often openly supported by liberal democracies. This was done by various means which ranged from simply condoning acts, disregarding or indifference to acts of state sponsored terrorism, to the supplying of weapons and training to the providing of military (particularly special forces troops) aid and intelligence advisor's for training, through to large scale deployment of military force. 

Examples of this would be US policy in central and South America through out the 1970's \citep{gareau2004state}, where everything from providing training (particularly through the use of the school of the America's) to the provision of advisor's and weapons \citep{koonings1999societies} could be labelled as acts supporting state terrorism. It should be remembered though that the use of labelling acts as state terrorism is extremely contentious, as it can often be attributed incorrectly and is often not considered terrorism. The GTD for instance does not consider state terrorism as a form of terrorism.
\item The new terrorism. This type of terrorist activity came about at the end of the the 20th Century. It is characterized by its wish to carry out large casualty type attacks, these groups are defined as having innovative command and control structure, a pan-national religious or political shared belief and their own moral norms for the justification of political violence. Al Qa'ida would be the prime example of new terrorism, with the attacks on September the 11th being the most publicized attack of this type \citep{burke2004qaeda}. Another well known example of such a group would be Shinrikyo cult, \citep{morgan2004origins} who carried out the Sarin gas attacks in Tokyo in the mid 1990's.
\item  Non-state terrorism or dissident terrorism. This is terrorism that is carried out by non-state actors against either the government or different ethnic groups. Examples of dissident terrorist groups would be the IRA in Ireland or ETA in Spain \citep{lutz2009successful}.
\item  Religious Terrorism. This can be defined as a type of terrorism where the group carrying out the terrorist acts believes they are being  allowed to do this through their faith in a higher being. Religious terrorism is often accompanied by literal interpretation and a strict adherence to a persons holy book \citep{pratt2015terrorism}.
\item Ideological terrorism is a type of terrorism inspired by a belief system or political ideology. The central theme of this type of terrorism is a blind belief in a political or religious ideology. On the political spectrum terrorists come from both the left and the right.
\item International terrorism are terrorist acts which have very clear global repercussions or impact in particular countries. These acts aim to focus world attention on a particular cause. One of the most prominent examples of this were the Munich Olympic attacks, the kidnapping and murder of Israeli athletes at the 1972 Munich Olympics would be one of the prime examples of international terrorism. The aim of international terrorism is not only to gain attention but also to gain both acknowledgement and compromise from the people they are in conflict with. Abu Iyad, the architect of the Munich attacks commenting on the attacks, concluded on the outcome of the Munich attacks: \textit{"The sacrifices made by the Munich heroes were not entirely in vain. They did not bring about the liberation of any of their comrades imprisoned in Israel ... but they did obtain the operation's other two objectives: World opinion was forced to take note of the Palestinian plight, and the Palestinian people imposed their presence on an international gathering that had sought to exclude them"} \citep{iyad1981my}.
\end{enumerate}

\section{Terrorist tactics}
Terrorists must be adaptive in their tactics \citep{bennett2007understanding} as the security apparatus  maintained by government can quickly implement countermeasures. A terrorist can deploy a number of types of attacks:
\begin{enumerate}
\item Arson. The legal definition of  Arson, defines it as the purposeful destruction of buildings, land or assets through the use of fire. Arson attacks are carried out against infrastructure necessary for the proper functioning of society. While arson attacks are less sensational than other types of attacks.  They are easy to carry out, requiring little technological know how, no access to advanced weaponry as the materials required to build an incendiary device are cheap and readily available. One of the most notable recent uses of arson would be the the attack on the US mission in Benghazi in 2012 by Ansar al-Shari`a and al-Qa`ida in the Islamic Maghreb (AQIM) to kill Ambassador Christopher Stevens \citep{maldonado2015leading}.
\item Assassination. This is the unexpected, audacious murder of persons who are strategically critical to a regime or group for doctrinal (both political or religious) reasons. Assassination does require a certain amount of technical expertise and may require access to advanced weapons and opportunity to get close to the target. The aim of assassinations are to strike at the heart of regime, instilling fear and chaos.  The use of assassination by al-Jama'a's terrorists in 1981 to murder President Sadat of Egypt while attending a military review remains one of the most striking uses of the tactic \citep{haykal1983autumn}. Another example of the use of assassination as a terrorist tactic to remove key leaders of opposition would be the murder of the Ahmad Shah Massoud, the leader of the United front, main opponents to the Taliban and AL Qa`ida regime in Afghanistan \citep{wolf2003assassination}. This had the effect of removing the only person capable of offering a credible alternative to the Taliban \citep{rashid2001fires} in Afghanistan.
\item Cyber. Cyber terrorism is the wilful and malevolent disturbance and upheaval of nationwide or global computer networks by an individual or a group of individuals. This is achieved through the use of computer viruses, DDOS (Distributed denial of service) attacks, computer hacking and damage to critical infrastructure \citep{Golandsky2016}. High value targets  such as high profile commercial buildings or developments, historic or government buildings and buildings of religious significance are targetted. While the potential of cyber terrorism has received considerable attention and caused much consternation, very few instances of cyber terrorism have been recorded and its use has been limited to attacking websites and communication networks. For example Hezbollah were able to hack into the Israeli Defence Forces secured mobile network "Vered Harim" in 2006 \citep{Golandsky2016}. Islamic Jihad have also developed specialist software to gain access to CCTV systems and track aircraft. However successful attempts to gain access to critical defence or public infrastructure have been limited, though in 2008 Russian sponsored attack on the Ukranian power grid \citep{Cybersecurity2016Perez} occurred. The benefits to using this type of terrorist attacks are that it is relatively easy to carry out and inexpensive, though does require considerable technical expertise. 
\item Economic attacks. These are attacks designed to cause financial hardship or loss and adverse economic affects \citep{drake1998role}. For example terrorist attacks against ports and shipping cause the introduction of extra costs to pay for increased security and safety. Piracy can also be used to increase be as a tactic of economic terrorism, causing shipping companies and governments to increase spending in insuring maritime safety and also the diversion of military forces to protect shipping. This can lead to an increase in cost of importing and exporting goods. 
\item Environmental. Environmental terrorism is the purposeful introduction of toxic or hazardous material into the environment with the aim of causing pollution. An example of this would be the poisoning of a countries water supply \citep{gleick2006water}.
\item Explosives. Through the use of conventional explosives such as RDX (a well known commercial military formulation is Semtex) and HMX sourced commercially has been used widely to build IED's (Improvised explosive devices) \citep{kopp2008technology}. Alternatively explosives can be acquired through the manufacture of explosives. Homemade explosives used by terrorists are often based upon inorganic salts and/or peroxides and can be synthesized from legally acquired chemicals, whose purchase would not cause suspicion \citep{johns2008identification}.  An example of usage of such compounds to such devastating affect where the Bali Bombings of 2006 \citep{royds2005case}.
\item Hijacking. Hijaacking is the illegal and illicit seizure of a mass transit vehicle by a group or persons with the aim of creating a mass media spectacle were the hostage situation can be used as a  mechanism to force concessions (release of prisoners) or elucidate money from a government or organisation. 

While aircraft hijacking is the most well known form of this type of attack, occasionally other mass transport types are targeted, notably ships with the seizure of the Achille Lauro \citep{halberstam1988terrorism}.

\item Hoaxes or threats are used to by terrorist groups who have established a past record and capability of carrying out attacks as a weapon of intimidation. Hoax's can serve to not only intimidate but by keeping security forces in an almost constant state of alert and deployed, degrades their effectiveness and readiness through exhaustion from being deployed for extended periods \citep{nagl2008us}.

\item Stochastic terrorism. Stochastic terrorism is the utilization of information media (previously television and radio but now the internet) to radicalise and exhort random members of the population to participate in terrorism. The stochastic terrorist is considered the instigator of the violence and the perpetrator of the violence is considered to be the lone wolf. The term stochastic is used because the acts may be statistically predictable but individually random. Stochastic terrorism can not only be used to trigger individuals with views sympathetic to the terrorists world view but also people suffering from mental illness, personality disorders etc. In this way it makes the use of targeted surveillance extremely difficult. The idea of stochastic terrorism is troublesome, concerning what role certain media played in a persons radicalization, as how do we know what specific method of mass communique triggered the terrorist attack. This has fundamental effects for civil liberties, as if dissemination of radical media is leading to triggering of terrorist attacks, then it would make sense to control this media, but this would violate freedom of speech and thought. Also to what extent the media had in triggering the attack as compared to the mental well being of the person is difficult to ascertain. 
\end{enumerate}

\section{Does terrorism work?}
In Alan Dershowitz's important work, \textit{Why terrorism works}, he argues that the advancement of the Palestinian cause since the early 1970's \citep{dershowitz2002terrorism} suggests that terrorism does work and is therefore a completely rational tactic to opt for or employ to accomplish a political aim.  Numerous case studies and studies based on game theoretic model have cited the example of Hezbollah whose sustained terrorist actions (most notable of these the 1983 attack on the US Marine barracks in Beirut, Lebanon) after the US and French deployment of peace keeping troops in the early 1980's during the Lebanese civil war, forced their withdrawl \citep{atran2004trends}. 

A more recent example of this phenomenon would be the 2004 Madrid bombings, which resulted in Spain withdrawing its forces from Iraq \citep{rose2007does}. These studies have suffered from focussing on particular cases and being selective in nature, lacking the scope to establish a more universal truth (they are difficult to generalize).
\\
While terrorism has often been studied using case studies or game theoretic's, very few studies have taken a data-centric approach focussing on outcomes of terrorist activities. One of the first such empirical studies, carried out by the US state department found that rarely did terrorist groups achieve all their aims (approximately only 7 percent of the time). The tactic of terrorism is strongly correlated with a group not meeting its objectives \citep{cronin2004foreign}. In Abrams empirical study on whether terrorism achieved its goals, very little evidence was obtained to support the hypothesis that terrorism was an effective strategy, infact the opposite was found, very rarely did terrorism achieve its goals. \citep{abrahms2006terrorism}.

\section{Countering terrorism and countering the terrorist narrative}

Counter terrorism is the unified application of theory, practice, specific military techniques and approach (including specialist units), government strategy which can be deployed to impede and eventually defeat terrorism \citep{jackson2005writing}. Just as terrorism tactics can take many different forms so can counter terrorist strategies. From a top down view an anti-terrorist strategy starts with a policy direction dictated by a government. This policy dictates the (political) direction and how the counter terrorist policy will be enacted as laws and what powers these laws will give military, police and (para military) intelligence agencies in combating terrorism. 

They may also bring about structural changes to the forces who carry out counter terrorism by changing the structure of these entities, changing the focus of the organisation or merging or forcing more collaborative work \citep{jamwal2003counter}. Governments may also pioneer the use of new technology in countering terrorism, one of the most visible of these technological developments, was the development of mass data collection and data mining to help process the information for investigation that emanated from the war on terror. Finally governments will also deploy military, police or para military (the meaning of the use of paramilitary forces refers to intelligence operatives) forces against terrorist which may be either domestic or in some cases (the war on terror, the Israeli response to the Munich terror attacks) foreign \citep{calahan1995countering}. The type of military / para military / police actions taken by a government range in their scale and breadth. These operations can range from surveillance, propaganda, arrests and raids on terrorist locations with the aim of interrupting their activities, large scale deployment of troops, assassinations or targeted killings upto full scale invasions of territories where terrorists operate from \citep{conetta2002strange}. 
\\
Governments may also seek diplomatic solutions to terrorism, from building coalitions to fight international terrorism or 'The new terrorism', to changes in international law to make it easier to investigate or prosecute terrorism. Diplomatic solutions to the prosecution of terrorism often cause liberal western democracies particular problems as they require cooperation with states they would otherwise be strongly opposed to \citep{jarvis2014critical}. An example of this would be Saudi Arabia, a country whose citizens have provided much material and human capital to Al Qa'ida \citep{abuza2003funding}. They may also require cooperation with groups who could themselves be regarded as terrorists. This has the effect of opening these governments up to criticisms of hypocrisy or imperialism as they are allying themselves with people who are themselves despotic or are seen as an invading alien (from a cultural and political point of view) force. During the Colombian response to FARC terrorism in the 1980's and 1990's  the war on FARC by government police, paramilitary and military forces was supplemented with right wing anti-communist militias. These groups can aid governments through the provision of additional manpower. However, Groups such as the United self defence forces were responsible for a number of vicious campaigns against FARC forces \citep{ColombiaRightWingTerror}. They were also responsible for the killing of many civilians and people whose ideologies they opposed particularly trade unionists, communists and democratic socialists \citep{peceny2006farc}. Such groups actions can act as a obstacle to peace with terrorist groups unwilling to negotiate with governments while such groups exist \citep{ColombiaAUCFARC}.
\\
In circumstances were force is used, it is important that it is used as judiciously as possible. The rules of international law should be followed, while at an operational level certain practices may be unique when compared to war \citep{roberts2002counter}. A measured response is required with civilian casualties being kept to a minimum. Over zealous counter terrorist responses can also act as a propaganda tool for the terrorist themselves (as they are seen as an opposing force to perceived imperialists or forces who are allied with undemocratic regimes), by framing the counter terrorist activities as being a disproportionate use of force especially when civilian casualties are high. However minimization of casualties can sometimes be difficult especially when countering terrorists operating in densely populated urban areas \citep{graham2009urban}.  
\\
Another concern for governments is how they frame a counter terrorist strategy. For instance the US response to the attacks on September the 11th 2001 was to declare a 'war on terror'. This terminology frames the terrorist campaign instigated against US which was tantamount to mass murder, as not a criminal act but as an act of war and the terrorists as not criminals but as soldiers. This legitimization of terrorism serves to further the cause of the terrorists through dignifying it and the terrorists as combatants \citep{moeller2009packaging}. One successful counter terrorism  strategy employed by governments is that based upon the British experience in Malaysia and (eventually) the troubles in Northern Ireland, that is that terrorism is a political problem and requires a political solution. This was set out in a counter terrorism framework by Robert Thompson \citep{hamilton1998art} who created five fundamentals of a counter terrorism \citep{thompson1966defeating}, these being:
\begin{enumerate}
\item The government has a well defined political goal.
\item The governments intelligence services, police and military forces must operate within the bounds of the law.
\item The government, it's intelligence services, police and military forces have a clear strategic framework for defeating terrorism with all actors working together to achieve this and not in competition with each other.
\item The governments focus must be on collapsing the terrorists attempted disruption and overthrow of the political system and not defeating the group militarily.
\item A government must first secure its hinterland or initial area of operation before moving into the areas affected by terrorism or insurgency.

While this strategy proved successful in Malaysia, a successive British administration in dealing with the troubles in Northern Ireland, initially followed almost exclusively a military strategy before moving to a more successful strategy modelled on Thompson's doctrine \citep{lafree2009impact}. During the 1970's IRA (British) mainland bombing campaign, the 1974 prevention of terrorism act introduced the term 'suspect community' by directing  laws targeting specifically the Irish community \citep{hillyard1993suspect}, a direct contravening of the counter terror axioms developed by Thompson. This only served to make the Irish community in Britain more vociferous in their support of the PIRA and more removed from the British political system. It is also been argued that the Muslim community in Britain since September the 11th has also been under the law, categorized as suspect, leading to discontent and isolation of the Muslim community and creating the environment for radicalization of the Muslim community \citep{pantazis2009old}.
\end{enumerate}
This argument has been countered by the narrative that "the new terrorism" is a more severe form of terrorism and is in-fact becoming more warlike, being capable of destroying western liberal states, it is important that this type of terrorism be fought on a war footing, requiring more severe responses \citep{bobbitt2008terror}.

Governments face a challenge in structurally changing their intelligence services and military to fight a counter terrorist campaign instead of a more traditional and conventional conflict. Often the technology they have put in place for use by their military is not fit for purpose and neither is the structure of the military whose forces are setup to fight a conventional war and not a counter terrorist campaign \citep{gazette1989changing}. This position can be further complicated as a government must maintain a significant conventional force as a deterrent to conventional wars and in case an actual conventional conflict breaks out \citep{gates2009balanced}.

\section{Research Question to be addressed in the study}
Businesses (particularly insurance), governments and risk management professionals face particular challenges with regard to terrorism. Being able to diagnose changes in intensity of terrorism is crucial so as to take action to take account (by countering it) for this change in behaviour. The most obvious changes in intensity would be increased occurrences of terrorist incidents or deaths due to terrorism. Other changes would be for instance a change in the attack vector (i.e. assassination, bombing, attacks on infrastructure) or weapon type (i.e. biological, chemical, explosives). Others would be has there been a change in geographic location of attacks (or what is the current geographical base of attacks is there a rural urban divide). 

These changes in behaviours may then prompt a different response from business or government. For example, for the purpose of insurance, an insurer may want to be alerted as soon as a change an increase in deaths due to terrorism or terrorist incidents is detected so it can account for the higher risk associated with the increase in terrorism, by increasing insurance costs. However it would also want to be sure that there has been a change in behaviour and not just a 'one-off' anomalous attack. For a government, a behavioural change in terrorism (due to change in incidents or deaths due to terrorism) would provide an impetus to carry out an intervention to counteract the change. Governments may also want to examine the effects of a specific intervention and has it had the desired effects (i.e. a reduction or stabilization in levels of terrorism in terms of either events or deaths due to terrorism). 

The benefits of using an electronic terrorist incident database is that they expand the range of studies that can be carried out to determine whether a change in behaviour with regards to terrorism has taken place. Terrorist incident databases also ensure the data is of higher quality and reliability as it has already been processed to improve quality and classified (or encoded) under certain criteria to aid in its use. They also offer the benefit in that they are easily accessible and can be accessed by businesses, governments or researchers free of charge. 

Particularly they allow the application of statistics and data mining techniques to the study of terrorism.

The research question directly addressed in this thesis is whether one can detect changes in behaviour associated with terrorism (utilising electronic incident databases). These would primarily be changes in the intensity of terrorism in terms of deaths or counts of incidents, but also have there been changes in terms of temporal, temporal spatial changes, changes in attack or weapon vector. Temporal spatial effects would be if there has been a change in the precedence of regions or countries in terms of terrorism over time. To examine whether changes in behaviour have occurred, a number of modelling techniques which are appropriate to modelling count and time series count data are utilized. The methods specifically used to model the count of deaths and incidents in Iraq are count regression modelling techniques and count time series analysis. The latter techniques are used to identify 'interesting' time count events, which can take the form of mean or gradual shifts, time series outliers or outbreaks (either using a E-Divisive with Means or syndromic surveillance based methods) or 'epochs of high and low terrorism' and the probability of being in one or the other, detected using Hidden Markov Models (HMM's).      

\section{Thesis overview}
The thesis will take the form of:
\begin{enumerate}
\item Chapter 1, serves as an introduction to the topic of terrorism and includes the following; what defines terrorism, terrorism in an historical context, what is counter terrorism and how one counters terrorism. This chapter also details the particular research question addressed in the thesis, can one detect changes in behaviour in terrorism using the terrorist incident database. 
\item Chapter 2, will be an introduction to data and data mining approaches to research in terrorism and in investigation of terrorism along with its application to counter terrorism. The chapter also covers the failures and challenges faced concerning the use data mining when applied to terrorism or counter terrorism research.
\item Chapter 3, will be an introduction to electronic terrorism databases, open and closed and their use in analysis of terrorist incidents. The chapter also covers the use of data held within the database, how the data is encoded and what problems arise when using the GTD. A review of the use of the GTD in terrorism research is also carried out.
\item Chapter 4, will address the application of machine learning techniques to the investigation of terrorism incidents. Both Chapter 4 and (appendix A)  address the specific research question at the centre of this thesis can one detect a change in behaviour (in terms of incidents or deaths) of terrorism associated with a particular country. In this chapter a number of methods appropriate to the analysis of time series count data are applied to the time series dataset of post invasion count of deaths in Iraq created from the GTD. A number of algorithms are applied to the dataset, two of these are online time series detection algorithms (the twitter outbreak detection algorithm and Netflix's SURUS algorithm) and two medical syndromic surveillance algorithms (EARSC based method and Farrington's method). 

Application of outbreak detection algorithms using purrr over multiple countries is also demonstrated.
\item Chapter 5 will serve as conclusion to the study, detailing both the usefulness of data mining techniques but also any insights uncovered from the analysis.
\item Appendix A, contains an exploratory analysis of electronic terrorism databases and ancillary datasets such as polity, transparency international or Fas.org. A number of steps were carried out in this section including exploratory analysis using visualizations, generation of summary statistical data and exploratory analysis. Other steps in this section deal with cleaning and validating data and the methodology used through the study (CRISP-DM). The exploratory analysis carried out in this chapter was directed with the motives of gaining an understanding of both the data held in GTD, to uncover any changes in behaviour in terms of terrorism on both a global and regional level and to discover appropriate techniques for modelling the data for the purpose of being able to detect changes in behaviour. Preliminary modelling of the data is carried out using count regression modelling and time series count data with HMM's. The output of these models use is discussed along with problems with their usage particularly model specification (in case of count regression modelling) and difficulties in generalizing the models (in the case of HMM's).
\item Appendix B contains preliminary modelling regression coefficients and contingency table s supporting the CA used in the preliminary modelling.
\item Appendix C contains outbreak and outlier detected count time series aberrations.
\item Contains details of the code repositories used during the research.
\end{enumerate}

\section{Discussion}

Terrorism while often being portrayed as a recent phenomenon, it has ancient roots demonstrated by presence of terrorist groups such as the Sicarii and the Assassins. While having its roots in ancient times, the tactic of terrorism became highly popularized firstly by national liberation movements and left wing groups in the second half of the 20th century. Then by religious / pan nationalist groups (in the early 21st century) which are now the most common perpetrators of terrorism.  Terrorism is most commonly defined as the use or the threat of violence against the populace, the agents of the government and government with the aim of advancing their political/ideological/religious beliefs.

Another key component of terrorism is that it is carried out by 'non state' actors. This last point can be problematic as the use of terrorism by government agents such as sections of the security forces or government backed paramilitary forces is a reality. This second attribute of terrorism has proved often to be a point of contention with academics, with many of them disagreeing with this second component. Many examples exist of state sponsored terrorism from the Nazi's use of terrorism in occupied countries in WW2, to the use of state sponsored paramilitary forces to terrorize the populace in central America in the  1980's, exemplified by the murder of Arch Bishop Romero of San Salvador in the 1980's \citep{romero1998violence}. 

While state terrorism is one particular form of terrorism (or in the strictest sense of the word not) other forms of terrorism include religious terrorism, dissident terrorism, idealogical and international terrorism, but the most prescient of these types of terrorism is what is known as 'new terrorism'. New terrorism has become the most prevalent form of terrorism and is exemplified by attacks which create the most casualties and are carried out by groups which are characterized as being pan national and/or religious, such as Al-Qa'ida or the Shinrikyo cult. It also employs innovative command and control structures and particularly innovative communication techniques. 

New terrorism has become strongly associated with the tactic of stochastic terrorism a term used to define the way in which mass media methods utilizing the internet like youtube, facebook, twitter, online manuals are used to disseminate their methods, their core message and their attacks with the hope of encouraging random people to carry out terrorist acts inspired by the material \citep{margulies2016trump}. 

Terrorism can be seen to have a number of over-arching tenets besides the use of violence to coerce a change and it being carried out by non-state actors, these are by the standards of international law as laid out by the Geneva or Hague conventions, they are illegal. However to the terrorist they are considered to be a form of 'just war' being permitted through either a religious, political or ideological perspective.

When looking at why terrorism occurs it is often wrongly argued that a legitimate grievance must exist for the conditions to be created for a terrorist group to flourish. This is not always true often terrorist groups are constituted from wealthy, well educated middle class and young disaffected people who see the only chance of affecting political change through terrorism. Examples of this behaviour would be the leadership of the red brigades in the 1970's in Europe and Japan to the leadership of Al Qa'ida who emerged and flourished throughout the 1990's. Both Al Zawahiri (Al Qa'ida's operational brains behind the attacks on the world trade centre) and Osama Bin Laden (the leader of Al Qa'ida at the time) were both from wealthy middle class backgrounds \citep{henzel2005origins}.

The strategic and tactical aims of terrorism are disparate and diverse. The strategic aims of how a group attains a political or ideological change against a government is achieved through many strategic aims. These range from attrition by making the cost of continuing to oppose the terrorism impractical and inordinate, intimidation where the terrorist aim to subjugate the populace to its cause by demonstrating an ability to punish lack of perceived loyalty of the populace and also to show a lack of ability on the governing bodies part to protect its citizens. 

Another common strategic aim of terrorism is provocation which is the use of terrorist acts to provoke actions against the populace. Some of the most evident examples of this would be the bloody Sunday killings by the British army in Derry in 1970 where a overtly heavy handed response to a peace March in Derry which ended in the killings of 14 people \citep{dawson2005trauma}. The subsequent invasions of Afghanistan and Iraq following the September the 11th attacks could also be argued to have been an example of a provocation strategy.  The last two strategies associated with terrorism  can largely be seen as competitive strategies amongst terrorist groups to establish themselves as the dominant group, these are outbidding and and spoiling. 

Spoiling is where groups try to dissuade dialogues between rival terrorist groups and the government, while outbidding is where the terrorist group tries to demonstrate it being the dominant force of opposition to the government by carrying out more and more audacious attacks. Examples of spoiling would be the continuity IRA's terrorist campaign post the peace process to try and re-ignite the conflict in Northern Ireland \citep{whiting2015spoiling}. An example of the outbidding strategy would be the 'IN AMENAS' attack by a Mokhtar Belmokhtar led group of terrorists who broke away from Al Qa'ida on the Tigantourine gas refinery, which was carried out by Belmokhtar and his followers had become disillusioned with AlQa'ida's campaign in North Africa and wanted to up the ante \citep{watling2015ctc}. 
The tactics employed by terrorism are also diverse ranging from more traditional methods such as arson, assassination, hijacking of mass transit vehicles to more modern forms of terrorism such as environmental to cyber attacks. Cyber terrorism is of particular concern to governments as not only can it damage critical infrastructure it can also undermine confidence of the public in the government \citep{gross2016cyber}. It is also difficult to counter and cyber terrorist operations are inexpensive to launch though do require technical expertise \citep{blakemore2016policing}.
Why terrorism tactics and strategies are important to understand is that they inform both the counter terrorist approach at an operational level and at a governmental level.  For a government this may involve making a political, military intervention to remediate against the particular terrorist tactic. 
At the key of any counter terrorist strategy is government policy which will direct and inform all counter terrorist activities and whether they be military/paramilitary or political in nature. Successful frameworks as laid down by successful practitioners such as Robert Thompson dictate that a government must have a clear political goal, it must operate within the bounds of the law, its focus should be focussed disrupting the terrorist groups operations and particularly winning support of the local people. 

It also requires securing first the hinterland before moving into areas affected by terrorism. However this strategy has strongly been contested especially in the light of what is referred to as 'new terrorism', defined by being a more extreme, more violent and having a more advanced command and control structure (which is characterized as being loosely interconnected and having very limited command and control structure) and employ 'stochastic terrorist' methods. Such an extreme form of terrorism requires a change in the way governments counter the threat posed from terrorism, this necessitates governments to view terrorism not as a criminal act but as war. This compels government to make changes to its military and security apparatus and calls for enhanced inter-agency and multinational cooperation between governments. Central to a counter terrorism strategy is an effective counter cyber offensive measure (particularly countering their communications networks or methods of disseminating information), better terrorism informatics, air power and the use of special forces troops and a move away from more conventional forces. \citep{lesser1999countering}. However this strategy has evolved, \citep{lesser1999countering} to a more wide ranging strategy that encompasses strengthening and deepening deterrence, limiting the likelihood of the use of weapons of mass destruction to avoid acts of terrorism of unprecedented magnitude and a number of other initiatives, which include: 
\begin{itemize}
\item Increase the capacity and willingness of states to act against terrorism, when it occurs. 
\item To place a microscope on terrorist groups and show what they are doing and how it affects the population. 
\item To decrease the amount of chaos and instability that the terrorist base is located in (to try and change failed states to be more functioning and especially embrace the rule of law). To codify counter terrorism strategies as part of formal international treaties and alliances. 
\item To limit a countries exposure to terrorism by limiting as much as possible by hardening a countries infrastructure to possible effects of terrorism, more controversial it may also require governments to target non state actors and individuals.
\end{itemize}

\section{Conclusion}
Terrorism is a difficult subject to define with no universally agreed definition of terrorism. It does have traits or tenets which define it, these include; the utilization of violence to gain an ideological gain whether that be political, religious or ideological aim. It is carried out by non-state actors (though they can be state sponsored in the case of state sponsored terrorism) and its aim is to influence the public as well as the government. While the conditions for the rise of terrorist groups has often been wrongly argued to be that of a strongly disaffected ethnic or politico-economic minorities, this is not always the case as is the example of the red brigades of Europe and Japan of the 1970's and 1980's. Groups carry out terrorist acts for a number of reasons; to intimidate the populace, to inspire the populace (to rise against the current ruling entity) by attacking the government. 

Terrorism also serves to undermine the government by attempting to disrupt its supply of critical services to the populace or to cause economic effects by discouraging DFI (Direct Foreign Investment) there by underpinning its ability to govern. These acts while at tactical level do contribute to the strategic aims of terrorism these include attrition, intimidation, spoiling, provocation and outbidding.

Terrorism is also diverse in the number of vectors that are employed to carry out acts of terrorism, these include, Arson, assassination, cyber-terrorism, environmental, hijacking, use of explosives, hoaxes or threats and economic attacks.

Counter terrorism efforts are techniques used to impede and eventually defeat terrorism, a counter terrorist strategy is policy directed, by a government, that dictates both the direction of counter terrorism policy and the tactics of how the counter terrorist policy will be enacted. A number of successful counter terrorist strategies have been proposed and been shown to work and frame works have been created for their implementation. The most notable of these are those of Robert Thompson, who employed them in the successful Malaysia campaign against Maoist terrorists in the 1950's and 1960's. Key to Thompson's framework was the implementation of a well defined political goal, the operation of all state actors (military, police etc.) within the bound of the law, the securing of an initial base of operations in the hinterland before expanding operations to areas outside this.

The key objectives of the research to be carried out in this thesis is to be able to detect changes in behaviour regarding terrorism these changes may be changes in intensity of attacks, changes in attack vector weapon type or geographical location of attacks ( for example whether there is a rural urban divide). For example, changes in intensity of attacks are monitored by detecting changes in counts of deaths or incidents and being able to classify the type of change detected, whether it is a persistent change, or is it signalling an outbreak or a non persistent anomaly such as a one-off spike or dip in terrorist attacks. Being able to detect such a change would be advantageous to businesses or governments as detection of such events could allow these entities to take remedial action to counter such changes. For a business such as an insurer this may involve accounting for these changes by increasing insurance rates once a change in behaviour has been detected, or even not offering insurance till a change in conditions (to a lower intensity) is detected.