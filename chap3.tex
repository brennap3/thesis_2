\chapter{Electronic terrorist incident databases}

\section{Introduction to terrorism incident databases}
Most of the initial research on terrorism was done by utilizing "small-n" qualitative case studies, with the emergence of large n-scale trans national database of terrorist events, these large n-scale databases allow "large-n" quantitative studies. 
Since the publication of Geddes's paper on how selection bias can influence your results. his has resulted in a large body of work directed at the standard to allow maximum levels of clarity, fidelity and substance in the data \citep{geddes1990cases}. At the same time there has been a realization that "large-n" studies involving  data on political incidents such as terrorism suffer from the same problems that curse "small-n" studies.
 
Terrorism event databases are standardized analytical record sets which are mostly sourced from media articles. These databases can be joined with other data sources such as world economic forum (WEF) to examine the causes and repercussions of terrorism. through their existence terrorism incident databases have been utilized to carry out time series analysis of terrorist incidents to see how effect tourism, foreign investment. But also they have been used to show the effects of particular interventions such as how the use of metal detectors in airports had an effect on hijacking of planes and also how increasing the the toughness of police responses on the level of violence in Ireland.

The five best known terrorism event databases are:
\begin{enumerate}
\item  International Terrorism: Attributes of Terrorist Events  ITERATE.
\item Rand Database of Worldwide Terrorism Incidents ,RDWTI.
\item Global Terrorism Database, GTD.
\item World Incident Tracking System, WITS.
\item Terrorism in Western Europe: Events Data, TWEED.
\end{enumerate}

One of the main benefits with the collection of terrorist incidents in databases is that they make the hitherto application of statistics to the study of terrorism which has seldom been utilized in the past, now possible. \citep{silke2001devil} carried out a survey of research carried out on terrorism between 1995 and 2000 and \citep{lum2006counter} audit of terrorism literature between 1971 and 2003 both concluded that only a tiny minority of research studies (approximately 3 percent) employed statistical analysis. This was much lower than allied research fields such as criminology. Criminologists have a wealth of official data available to the them including the FBI's UCR (Uniform Crime Reports) and census data which holds the NCVS (National Crime Victimization Survey).

In comparison to the collation of other criminal statistics the collection of information on terrorist activity is especially difficult. With typical criminal behaviour, a number of formal sources exist for recording this type of information. These include official police statistics, governments, news media (in open democracies) and  international organizations. Primary sources submitted by international organizations tasked with the gathering of criminal statistics include the International Police Organization (INTERPOL) \citep{bresler1992interpol}, United Nations (UN) criminal activity reports \citep{united2013global} and the World Health organization (WHO) who record homicide rates. Other secondary sources would include in the US, the national crime victimization survey and for an international form of the data. All these sources however fail to completely capture terrorist incidents or  are either completely omitted or fail to capture significant detail. While some governments do collect  information on acts of terrorism, such as the United States Department of State and the British Home Office \citep{Homeoffice2016}. However this data can often contain a large amount of bias due to political sensitivity of reporting terrorism. Secondly, while certain countries do produce  statistics on terrorism (such as those listed above) they are in the minority, and very few countries produce data around terrorist attacks. This has limited the research of terrorism to the collation of data from secondary sources primarily media sources, rather than on primary sources, however over time the collation of these secondary sources have led to the synthesis of ever increasing more complete and inclusive data sets for the study of terrorism.

Since the 1970's, terrorism databases have  began to evolve for the collation of terrorist incidents. The most notable of the is the Pinkerton Global Intelligence Services (PGIS) database \citep{dugan2006first}. This is not only the most well known of these terrorist incident databases, it also has the largest number of events recorded. PGIS database was an initiative started by the PGIS when they began to instruct analysts to recognize terrorist incidents and record them. The PGIS characterizes terrorism as "the threatened or actual use of illegal force to attain a political, economic, religious or social goal". The data collection process excluded insurgency activity (even when terrorist acts) and state sanctioned terrorist activities. This was due to the act that the databases primary function was to serve as a risk assessment tool for corporate clients. THE PGIS was assembled by specially trained analysts who were mainly ex US air force personnel \citep{fivethirtyeightGTD2015}, listening to news wire stories and recording the incidents in the database if they met the PGIS criteria of terrorism 

Also in the 1970's ITERATE (International Terrorism: Attributes of Terrorist Events) began to record terrorist incidents and classified the terrorist incidents according to a number of variables including;  
\begin{enumerate}
\item The date of the terrorist attack.
\item The country the attack occurred in.
\item The objective of the attack.
\item The nature of the terrorist attack carried out.
\item The number of casualties.
\item The identity of the terrorist group.
\item The origin of both terrorists and the victims.
\item A number of negotiation variables.
\end{enumerate}
ITERATE was again created through the analysis of news media. One particularly useful encoding held within the database was the negotiation variables proving highly useful in studies that involve the kidnapping of hostages \citep{GPOL:GPOL142}. The ITERATE  dataset which was initially worked on By Edward Mickolus \citep{mickolus2013iterate}. ITERATE has been the most extensively utilized data source for research in terrorism. While an international database , the data only records incidents from  1968-2000. The RAND corporation also collates a international (originally they planned to collect data upto 1998) terrorist dataset, which is created and maintained in conjunction with the Oklahoma Institute for the prevention of terrorism to create an additional dataset from 1998 on \citep{lafree2007introducing}.  The RAND database records transnational terrorism from 1968-2009 and domestic (US) terrorism from 1998-2009 \citep{sandler2013analytical}. Th ITERATE database has been shown to offer a higher coverage of coded variables than the RAND database and also has a wider scope of coverage transnational than the RAND database \citep{enders2011political}. TWEED is limited to acts of terrorism carried out in Western and Europe and is the only database created and maintained outside the US. TWEED  is also limited to to terrorist acts that originated from groups that are based in Western Europe and not imported from elsewhere \citep{engene2007five}. The TWEED database is also different from the other databases in that it is sourced from only source (as compared to the others, which are sourced from the news media), '\textit{Keesings Record of world Events}' \citep{east2016keesing}.

\section{The synthesis of the Global Terrorism Databases, the data collection process}
\label{sec:synglobalter}
University of Maryland through the Global terrorism Database initiative acquired the PGIS data and began not only digitizing these records but also began to qualitatively check the entries against the cases recorded in the RAND and ITERATE databases. Data collection beyond 1997 is carried out by the University of Maryland through sponsorship by the Department of Homeland Security (DHS) \citep{lafree2011building}. Originally the data collection for the GTD2 is currently conducted by the START \citep{startGTD2016} consortium at the University of Maryland and is led by the Start consortium. The database post 1998 is referred to as GTD2 \citep{lafree2010global}.The GTD team then assembled the GTD in a similar manner to the PGIS database and use a team of analysis who are fluent in a number of languages including Arabic, Spanish, Mandarin etc. These analysts then began analysing data in open sources of data such as Lexis Nexis (Professional) and Opensource.gov. These were then reviewed and those that are deemed to be terrorist acts are then added to the supervisors for review and then on review are added to the database, unless they are borderline. in this case they are referred to a criteria council to submit a verdict on whether they should be added to the database or not \citep{lafree2012generating}.

The collection process has now been somewhat revised, the analysts now use an advanced boolean search algorithm to filter the number of articles. The articles are now also sourced through the MetaBase supplemented with information from Opensource.gov. These articles are then processed using both NLP and machine learning to narrow the number of articls to review by the analyst.A further algorithm is then run on the data to remove equivalent entries. After this a machine learning algorithm  based upon a pattern recognition algorithm is the used to decide the likelihood of the news story being related to terrorism \citep{fivethirtyeightGTD2015}. The stories are then analysed by the GTD team and are then sorted manually and coded as described above \citep{ben2016pattern}. The pattern recognition is based on the clustering method utilizing the ${Khi}^2$ distance derived from multiple correspondence analysis. Multiple correspondence analysis is a development of correspondence analysis which allows the investigation of the relationships and affinity of a number of different categorical variables. Multiple correspondence analysis is achieved by carrying out a regular correspondence analysis on an indicator matrix. The percentage of variance that can be accounted for (by the different components) are then corrected and further adaptation of the distance between points is carried out. This can be explained in the following manner. Taking K number of variables and each variable has $J_{k}$ number of levels and the total of the levels ($J_{k}$) is equal to J. As we have I number of  observations, the I x J indicator matrix is expressed as X. Carrying out correspondence analysis on the indicator matrix provides two groups of factor score, one which corresponds to the rows and the other corresponds to the columns, these factor scores are then scaled in such a manner so that the variance equals their corresponding eigenvalues. The complete total of the table made of the rows and columns described above is expressed as N, from this the probability matrix (Z) can be calculated as  $Z = N^{-1}X$. r the vector of row totals of the probablity matrix is given by $r=Zl$ and c is the vector of column totals. The diaganol  matrices (for the row and column vectors) are given by Dc=diag(c) and Dr=diag(r). Through the use of the below singular value decomposition \ref{eq:eq1svd}. 

\begin{equation} D_{r}^{-1/2}(Z-rc^{T})D_{c}^{-1/2}=P\Delta Q^{T}  \label{eq:eq1svd} \end{equation}

The principal coordinates of the rows and columns as they pertain to the principal axis can then be derived as \ref{eq:eq2svd}:   

\begin{equation}  F=D_{r}^{-1/2}P\Delta  \quad \textrm{and} \quad G=D_{c}^{-1/2}Q\Delta \label{eq:eq2svd} \end{equation}

Where $\Delta$ is the diagonal matrix of the singular values and $\Lambda = {\Delta}^2$ is the matrix of the eigenvalues.

The adapted ${Khi}^2$ formula is shown below by the formula \ref{eq:eq1khisq}.

\begin{equation} D^{2}(x,x')=\frac{1}{p}\sum^{\alpha}_{\mu =1} \frac{(x_{i}-x'_{i})2}{m_{\mu /n}} = \frac{n}{p}\sum^{\alpha}_{\mu =1}\frac{(x_{i}-x'_{i})2}{m_{\mu }} \label{eq:eq1khisq} \end{equation}

Where:
\begin{itemize}
\item[] x and x' are two different observations from the dataset
\item[] p is the amount of data dimensions in each row
\item[] $m_{\mu}$ is the number of times the modality $\mu$
\item[] $\alpha$ is the amount of modalities present in each of the dimensions.
\end{itemize}

The $Khi^{2}$ distance was tested in conjunction with Euclidean distance and the $Khi^{2}$ distance measurement was found to be superior, delivering more distinguishable results.

\section{Measurement issues with using incident terrorist databases} \label{sec:markermeasdif}
One of the main drivers for the increase in utilization of statistics to study data is the rise of open source data sources on terrorism, which can be downloaded via the network. However quantitative terrorism involving these open source widely available datasets often suffer from problems, particularly the failure of most studies to utilize control groups. An additional measurement problem that faces researchers using open source data sources is the issue of source type reliability. The support materials used to add an incident  will often contain  inconsistent details. \citep{ackerman2016speaking} have also raised issues related to the legitimacy and authenticity of the open sources used to gather the incidents from. \citep{ackerman2016speaking} propose the inclusion of validity and credibility metrics for  open sources used to gather the terrorist incident information from. \citep{ackerman2016speaking} contest that enabling  an unambiguous analyses of the terrorist incident database. Related to this problem is coder reliability which is due to the use of machine learning and different people to encode variables, thus testing the validity of encoding if items in these incident databases is a problem which must be addressed when using these terrorist incident databases. 
Other problems associated with the use of terrorist incident databases are that of missing values, though some open source databases have had a large amount of success in overcoming this problem, not by imputing missing data but by carrying out additional directed  probes or explorations of publicly available information to limit missing data \citep{parkin2012developing}.


\section{The coding of the GTD database}
The data management system (DMS) employed by the GTD combines the functions of source material management and assessment with incident diagnosis and selection the subsequent process of incident coding and disseminating the information through the internet as a downloadable file. The tools developed for the GTD data collection makes this process seamless and allows each team to encode all the different types of attack.The coding strategy employed by the GTD involves one of the six teams in the generation of the GTD concentrating on one specific area. These include;
\begin{enumerate}
 \item Where the incident occurred.
 \item  The group or individual who carried out the attack.
 \item The target of the attacks.
 \item The weapons utilized in the incident by the attackers.
 \item The tactics used by the attackers.
 \item The number of people injured and killed and the ramifications of the attack.
 \end{enumerate} 
This methodology ensures that each piece of information is encoded by a team that is familiar with it, for example the team responsible for identification of the groups responsible for the attack  as this group due to working exclusively in this area will have greater knowledge working in this area, familiar with the relationships between different groups, naming conventions, name alternatives or aliases for different groups, factions an groups who have evolved from terrorist groups.
\subsection{GTD admittance metrics}
For inclusion into the GTD an incident is defined as a terrorist act if it meets certain criteria. While the GTD definition of terrorism is similar to the most widely held definitions of terrorism. The GTD data collection teams have imposed strict criteria for inclusion of incidents into the database, these are:
\begin{enumerate}
\item The event must have been intentional in nature.
\item The event must include some use or threat of violence.
\item The groups who carried the attacks must be sub-national in nature, terrorism carried out by states are not included.
\item The event must have been carried out with aim of achieving an economic, religious or social aim.
\item There must be some indication that the event was in the pursuit of sending a message to a wider audience than the immediate victims.
\item  The event must be considered outside the bounds or norms of rightful or what is deemed to be reasonable actions under the rules of war.
\end{enumerate}

Additional filters which may discount the act from inclusion are if it is part of a wider insurgency, it is a criminal act, it is part of an inter or intra power struggle.
Other problems with inclusion into the database of the same incident twice, so the analysts involved in collation of the incidents only record a single at specific location and time, if the location or time of the events are different they are counted as separate incidents.

\section{The database variables}

The database variables included in the database are described below . They can be grouped under the following areas:
\begin{enumerate}
\item GTDID (an id) and date, a unique identifier and year month date
\item Extended incident, did the incident last longer than 24 hours.
\item Incident information, a summary of the incident, what incident criteria the incident met to be included, is there a doubt the incident was terrorism and if there is doubt what is the alternative designation of the incident. 
\item Was it a component incident of an attack that was an multiple incident. What are the related incidents to the event.
\item The incident location, including country, region, province 
\item  Data related to the attack, including attack type, the success of the attack, whether the attack was a suicide attack.
\item Data on the weapons used in the attack, including weapon type.
\item The target or victim type including nationality.
\item Attacker information. The name of the group, number of attackers, number of perpetrators captured. Did the group claim responsibility and how was this claim made.
\item The number of casualties (including number or people killed and injured), the number of US injured and killed. How much property damage occurred.
\item Information regarding kidnapping or hostage taking. This includes the number of people kidnapped, the duration of time the kidnapping lasted, the origin country of the kidnappers, the resolution of the kidnapping, the amount of ransom paid, the ransom note details, the outcome of the event i.e rescue, attempted rescue, release of hostages etc.  
\item Additional notes on the event, containing miscellaneous information on whether coordination of a number of attacks within a specific locale and time. Unusual circumstances such as sudden changes in tactics.
\end{enumerate}

\section{Overview of electronic incident databases}
While Global incident terrorism databases have facilitated large scale n quantitative studies they do suffer from some problems. These include the conceptualization of what and what is not terrorism as well the encoding of these conceptualizations.  Greater transparency is also required around the context and sources of information used in the compilation of the database. This has the effect of increasing confidence a researcher has in utilizing the data. Some open source providers have made great strides in this
area particularly START in their synthesis of GTD who always quote citations for each event.Issues also exist around the coding of variables and how coding conflicts if they occur are resolved. The GTD serves as a good model for this using teams with specific expertise in a particular area to encode those specific variables. Other issues with the synthesis of a terrorist incident database is the inclusion  of variable indicating doubt in whether the incident was a terrorist event, while the WITS database includes a confidence variable around whether the event was a terrorist event.Other issues include the disclosure of those who backed/funded/payed for the research and who controls the information, this will allow the appraisal of any potential conflicts of interest that may occur as is in the case of the original PGIS, who discounted state terrorism due to the Pinkerton agency not believing logging these incidents would be of use to their corporate clients. 



 
