\chapter{Final discussion and conclusion}
\chaptermark{Final review and discussion}

\section{Detecting changes in intensity and behaviour in terrorism}
\label{sec:chap6intro}

Terrorism while often being portrayed as a recent phenomenon, occurrences have been present since antiquity. While having its roots in ancient times, it has recently come to prominence especially since the 1970's. Terrorism is most commonly defined as the use or the threat of violence against the populace, the agents of the government and government with the aim of advancing their political/ideological/religious beliefs.

Data mining has evolved as a powerful additional tool to the traditional intelligence tool-kit of signal intelligence (SIGINT) and human intelligence (HUMINT) and newer intelligence methods such as OSINT or CYBINT (particularly with these fields datamining can be viewed as an enabler of these technologies).

Data mining has been utilized to both study it's application to counter terrorism as well as its application to the study of terrorism itself. A number of different data mining types(supervised and unsupervised  methods) have been applied to both the study of terrorism and counter terrorism. An allied field to data mining for the study of terrorism is the field of terrorism informatics, which also includes the collation, collection and cleaning of data along with the presentation of the data in a simple to interact with and understand manner. Terrorism informatics systems may be also required to ingest data from many disparate sources, for instance IOT banks of sensors or social media data. Predicting acts of terrorism or terrorists though is an extremely difficult task from a  technical standpoint. 

While the prediction of the terrorist acts or classification of a suspect as a terrorist is extremely difficult due to the low signal of the actors involved in terrorism from the general populace (class imbalance) is particularly difficult. The application of data mining techniques to predicting terrorism due to the low occurrence of terrorists amongst the general populace results in high numbers of false positives (base rate fallacy) \citep{axelsson2000base}. However data mining and terrorist informatics are particularly useful in aiding the analysis of terrorism. 

Electronic terrorist incident databases have assisted terrorism research by enabling the use of quantitative research to terrorism and moving the study of terrorism away from small n qualitative studies (case studies) to large n quantitative studies. Terrorism incident databases have existed since the 1970's with the establishment of the PGIS \citep{enders2011domestic} and have morphed into electronic terrorist incident databases such as the GTD, which is open source, freely available, has clearly defined encoding standards, is regularly updated and maintained, utilizes open collection and collation methodologies and is freely disseminated through the internet. 

These types of databases are ideal for carrying out terrorism research for academic, economic and governmental motivations.

Being able to detect changes of intensity and particularly the type or classification of change (is an outlier, a mean shift etc.) is an extremely important task from both a political and economic viewpoint. From a social point of view early detection of changes of intensity in terrorism may influence a change in government anti terror policy or a military/para military response. From an economic viewpoint being able to detect the early onset of changes in intensity is equally important. For example being able to detect the early onset of increases in intensity of terrorism would be of particular interest to risk management professionals or insurers who would want to on the detection of changes in intensity of terrorism want to revise insurance costs upwards or withdraw from particular markets till such time as the risk due to terrorism has dissipated. Risk management agents may want to advice clients to withdraw employees from a particular country or region, till increases in intensity of terrorism subside or the situation stabilizes. 

In the field of terrorism research being able to detect changes in intensity are also of use especially in the fields of research of counter terrorism intervention analysis or backlash modelling, which are active fields of research which are aimed at discovering the outcomes of particular counter terrorism strategies on terrorism (did they affect an increase, decrease or neutral effect on the intensity of terrorism). 

\section{The study of statistical and data mining methodologies for the detection of changes in intensity of terrorism}

Initial investigation of detecting changes in intensity centred on using count regression techniques or HMM's. Both methods suffered from problems, the regression methods suffering from being incorrectly specified and the HMM's suffered from problems of generalizability and understandability. The problems found with the count regression models (Poisson, Quasi-Poisson, Negative Binomial) was that they suffered from incorrect specification. Using linear regression  and robust regression on log transformed also suffered from the same problem of incorrect specification. Also the data for the regression models had to be enriched by joining to datasets capturing coalition troop levels, major insurgent activity and US and Iraqi government detail (reign of Leaders of country). Due to the unspecified nature of the HMM's the correct number of transition states to model must be determined empirically. These problems of incorrect specification and lack of generalizability, make their application difficult across multiple countries or regions. 

Instead a number of time series count aberration techniques were tried to address the very weaknesses that were found in the preliminary modelling. One of the key learning outcomes from the research was that these techniques are not only highly generalizable, but are more pertinent to the specific research question being addressed (the detection in change in intensity of terrorism or resulting changes in behaviour). Four methods for detecting count time series aberrations were trialled, two syndromic surveillance methods (EarsC3, Farrington's method), EDM (SURUS) and outlier detection (RAD). These methods all detect different type of aberration:
\begin{itemize}
     \item SURUS, uses EDM and detects mean or gradual shifts.
     \item RAD, detects using PCA time series outliers.
     \item Syndromic surveillance methods detect values which exceed a threshold which would normally be expected. If this value is exceeded an outbreak is detected.
\end{itemize}     

These methods while valuable in their own right when used together can provide further classification of a count time-series aberration. Also they can be applied across multiple regions relatively easy using R's functional programming paradigms particularly purrr. These qualities make their use for the detection of 'interesting' count time-series events particularly useful. As far as can be determined none of these methods have been applied to the study of terrorism before and this makes their use for detecting changes in intensity of terrorism novel.

\section{Future work}

Aberration detection methods proved both easy to apply, highly interpretable and are highly generalizable when compared to typical methods for analysing count time series data. They are of particular use for detecting changes in intensity or behaviour in terrorism, with being particularly suited to the early detection of outbreaks or shifts in intensities of terrorism. 

Possible future work would be to apply the methodologies to the large n quantitative methods applied to backlash modelling/intervention analysis, which both study the effects of counter terrorist interventions. Backlash modelling is the study of counter terrorist initiatives having the opposite effect to the one desired by causing an increase in terrorism activity \citep{argomaniz2015examining}. While intervention analysis studies how successful a particular counter terrorist initiative was. A possible application of outbreak detection to backlash modelling/intervention analysis is to assess the success of a particular counter terrorist intervention would be to measure the number of outbreaks before and after a particular intervention. 
